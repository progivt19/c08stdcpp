%!TEX program = xelatex
\documentclass{article}
\usepackage[a5paper,hmargin=17mm,tmargin=15mm,bmargin=25mm]{geometry}

\usepackage{ifxetex}
\ifxetex
 \usepackage{fontspec}
 \setmainfont[Scale=1.1]{Arno Pro}
 \setmonofont[Scale=.92]{Consolas}
 \usepackage{unicode-math}              %% пакет для загрузки шрифтов математического режима 
 \setmathfont{[latinmodern-math.otf]}
 \setmathfont[range=\mathit/{latin,Latin}]{Arno Pro Italic}
 \setmathfont[range=up]{Arno Pro}
 \setmathfont[range=\mathup/{latin,Latin}]{Arno Pro}
\else
 \usepackage[utf8]{inputenc}
\fi
\usepackage[russian]{babel}
\usepackage{enumitem, graphicx, minted, microtype}
\usepackage[dvipsnames]{xcolor}

\usemintedstyle{vs}
\newcommand{\textex}[1]{\texttt{\color{ForestGreen}#1}}

\begin{document}
\renewcommand{\leq}{\leqslant}
\renewcommand{\geq}{\geqslant}
\noindent
\textbf{Лабораторная работа 8}\\
{\Large \textbf{Файловые потоки и~контейнеры STL}}\\
\strut\hfill\smash{\includegraphics[scale=0.2]{logo.png}}

Цель этой лабораторной работы~— познакомиться с~некоторыми объектами стандартной библиотеки C++.

Некоторые задания в~этой работе требуют ввода матриц. В~ходе написания решения вы будете несколько раз запускать программу для проверки. Если одно число еще можно повводить несколько раз, то повторно вводить $3\times4$ матрицу утомительно. В~таких случаях логично один раз записать данные в~текстовый файл, а потом заставить программу читать из файла. Как организовать ввод из файла и~вывод в~файл? 

В стандартной библиотеке Си++ предусмотрены объекты, представляющие текстовые потоки для ввода \texttt{ifstream} и~вывода \texttt{ofstream}. Они объявляются в~заголовочном файле \texttt{fstream}. Они должны быть связаны с~файлом на диске. Это можно сделать при создании объекта при помощи конструктора. Конструктор~— это специальная функция для создания объекта и~настройки его внутреннего состояния, возможно из некоторых начальных данных, передаваемых как параметры. При этом имя функции совпадает с~именем класса создаваемого объекта. Такое определение имеет вид
\trivlist
\item
  \strut\quad а) \texttt{{\color[rgb]{0.17,0.57,0.69}класс} имяобъекта = {\color[rgb]{0.17,0.57,0.69}класс}({\color[rgb]{0.64,0.08,0.08}параметрыконструктора});}
\item 
  или 
\item
  \strut\quad б) \texttt{{\color[rgb]{0.17,0.57,0.69}класс} имяобъекта({\color[rgb]{0.64,0.08,0.08}параметрыконструктора});}
\endtrivlist
Создадим объекты \texttt{in} и~\texttt{out} согласно варианту б), представляющие файлы \texttt{input.txt} и~\texttt{output.txt} для чтения и~записи, соответственно. Введем число из первого файла и~запишем строку, содержащую запись этого числа, в~выходной файл. Как видно, работа с~файловыми потоками не отличается от работы с~\texttt{cin}/\texttt{cout}. Хорошая практика~— закрывать файлы, как только они перестали быть вам нужны.
\begin{minted}{cpp}
#include <fstream>
using namespace std;
int main() {
  ifstream in("input.txt");
  ofstream out("output.txt");
  in >> n;
  in.close();
  out << "Hi " << n " times!\n";
  out.close();
  return 0;
}
\end{minted}

При работе с~файлами иногда бывает нужно считывать заранее неизвестное количество данных, пока файл не закончится. Потоковый объект типа \texttt{ifstream} <<не узнáет>>, что файл закончился, пока не попробует прочитать дальше конца файла. Ваша программа может проверить это, приводя такую переменную к логическому типу: \texttt{false} означает, что файл не в~порядке, в~частности, закончился. Таким образом, если известно, что в~файле \texttt{input.txt} записаны целые числа, но неизвестно заранее, сколько их, то узнать их количество можно так:

\begin{minted}{cpp}
#include <fstream>
#include <iostream>
int main() {
  std::ifstream in;
  int a, count = 0;
  in.open("input.txt");
  in >> a;
  while(in) {
    count++;
    in >> a;
  }
  in.close();
  std::cout << count << "\n";
  return 0;
}
\end{minted}






\newpage






\begin{center}
\textbf{ОБЩИЕ ЗАДАНИЯ}
\end{center}

\sloppy
\begin{enumerate}
\item
Дан текстовый файл \texttt{input.txt}, выведите на стандартное устройство вывода строку максимальной длины в~этом файле в~выходной файл \texttt{longest.txt}. Если срок максимальной длины несколько, вывести первую из них.
\item
Дан файл \texttt{input.txt}, содержащий целые числа по одному в~строке. Выведите на стандартное устройство вывода наибольшее из них, принадлежащее интервалу $[a, b]$. Целочисленные концы интервала $a$, $b$ вводятся со стандартного устройства ввода. Гарантируется, что в~файле присутствует хотя бы одно число из данного интервала.
% +1 вешников
\item
Дан файл \texttt{input.txt}, содержащий целые числа по одному в~строке. Выведите на стандартное устройство вывода только те из них, что не меньше последнего. Использовать векторы.
\item
Дан текстовый файл \texttt{input.txt}, содержащий целочисленную матрицу размера $n\times m$, записанную в~следующем виде. В~первой строке файла через пробел записаны два целых числа $n$ и~$m$, число строк и~столбцов в~матрице. В~каждой из последующих $n$ строк записаны по $m$ целых чисел, элементы матрицы. Выведите на стандартное устройство вывода максимальные элементы каждой строки поочередно, каждый в~отдельной строке.
\item
Даны текстовые файлы \texttt{a.txt} и~\texttt{b.txt}, содержащие две матрицы $A$ и~$B$, соответственно, записанные как в~предыдущем пункте, причем число столбцов в~$A$ совпадает с~числом строк в~$B$. Выведите в~файл \texttt{ab.txt} матрицу их произведения $AB$, в~том же формате. Использовать векторы.
\end{enumerate}







\newpage

\bigskip\sloppy

\noindent\centerline{\textbf{ИНДИВИДУАЛЬНЫЕ ЗАДАНИЯ}}

\medskip

{\color{red}\textbf{В задачах 1--2:}} Все матрицы, если не указано иное, являются целочисленными и~вводятся из входного файла \texttt{input.txt} следущим образом. На первой строке два числа $n$ и~$m$~— число строк и~столбцов, соответственно (либо одно число $n$, если матрица квадратная). В~последующих $n$ строках записаны строки матрицы. Размер матрицы по каждому из измерений не превышает 1000. Если не указано иное, в~качестве ответа следует вывести полученные матрицы/значения в~стандартный выходной поток.


\begin{enumerate}[label={}, leftmargin=0pt, itemindent=0pt]
\item
\begin{enumerate}[label=\arabic{enumi}.\arabic*.] % ------- #1 -------
  \item
  Дана матрица $n\times m$. Найти в~каждой строке матрицы максимальный и~минимальный элементы и~поменять их между собой. 
  \item
  Дана действительная квадратная матрица порядка~$n$ ($n$~— нечетное), все элементы которой различны. Найти наибольший элемент среди стоящих на побочной диагонали и~поменять его местами с~элементом, стоящим на пересечении главной и~побочной диагоналей.
\end{enumerate}

\item
\begin{enumerate}[label=\arabic{enumi}.\arabic*.] % ------- #2 -------
  \item
  Из файла \texttt{input.txt} ввести натуральное число $n$. Получить квадратную матрицу порядка~$n$:
  \[
  \left(
  \begin{array}{cccccc}
  0 & 0 & 0 & \ldots & 0 & 1\\
  . & . & . & \ldots & . & .\\
  0 & 0 & 1 & \ldots & 1 & 1\\
  0 & 1 & 1 & \ldots & 1 & 1\\
  1 & 1 & 1 & \ldots & 1 & 1\\
  \end{array}
  \right)
  \]
  \item
  Для заданной квадратной матрицы $A$ сформировать одномерный массив из элементов ее главной диагонали. Найти след $ \mathop{\mathrm{tr}} A$ матрицы, суммируя элементы полученного одномерного массива. Преобразовать исходную матрицу по правилу: четные строки разделить на полученное значение, нечетные оставить без изменения.
\end{enumerate}

\item
\begin{enumerate}[label=\arabic{enumi}.\arabic*.] % ------- #3 -------
  \item
  Из файла \texttt{input.txt} ввести натуральное число $n$. Получить квадратную матрицу порядка~$n$:
  \[
  \left(
  \begin{array}{cccccc}
    1 & 0 & 0 & \ldots & 0 & 1\\
    0 & 1 & 0 & \ldots & 1 & 0\\
    . & . & . & \ldots & . & .\\
    0 & 1 & 0 & \ldots & 1 & 0\\
    1 & 0 & 0 & \ldots & 0 & 1
  \end{array}
  \right)
  \]
  \item
  Квадратная матрица, симметричная относительно главной диагонали, задана своим верхним треугольником, выписанным в~виде одномерного массива: ввести из файла \texttt{input.txt} число $n$, затем $\frac{n(n+1)}{2}$ элементов данного массива. Восстановить исходную матрицу.\\
  Пример ввода:\\\textex{3\\1 5 0 4 2 -1}\\
  Правильный вывод:\\\textex{1 5 0\\ 5 4 2\\ 0 2 -1}
\end{enumerate}

\item
\begin{enumerate}[label=\arabic{enumi}.\arabic*.] % ------- #4 -------
  \item
  Дано действительное число $x$ и~натуральное число $n$. Получить действительную квадратную матрицу порядка~$n$ вида:
  \[
    \left(
    \begin{array}{ccccccc}
    x & x & x & \ldots & x & x & x\\
    x & 0 & 0 & \ldots & 0 & 0 & x\\
    x & 0 & 0 & \ldots & 0 & 0 & x\\
    \multicolumn{7}{c}{\dotfill}\\
    x & 0 & 0 & \ldots & 0 & 0 & x\\
    x & 0 & 0 & \ldots & 0 & 0 & x\\
    x & x & x & \ldots & x & x & x\\
    \end{array}
    \right)
    \]
  \item
  Дана квадратная матрица порядка~$n$, элементы которой~— цифры от 0 до 9. Найдите номер последнего по порядку столбца, в~котором содержится наибольшее количество различных цифр.
\end{enumerate}

\item
\begin{enumerate}[label=\arabic{enumi}.\arabic*.] % ------- #5 -------
  \item
  Из файла \texttt{input.txt} ввести натуральное число $n$. Получить квадратную матрицу порядка~$n$:
  \[
  \left(\begin{array}{cccccc}
  0 & 0 & 0 & 0 & \ldots & 1\\
  \multicolumn{6}{c}{\dotfill}\\
  0 & 0 & 0 & 1 & \ldots & n-3\\
  0 & 0 & 1 & 2 & \ldots & n-2\\
  0 & 1 & 2 & 3 & \ldots & n-1\\
  1 & 2 & 3 & 4 & \ldots & n
  \end{array}
  \right).
  \]
  \item
  Дана вещественная матрица $A$ порядка~$n$. Будем называть \emph{соседями} элемента матрицы $a_{ij}$ элементы матрицы $A$, у которых номер строки и~номер столбца отличаются, соответственно, от $i$ и~$j$ не более чем на единицу. Постройте матрицу $B$, состоящую из нулей и~единиц, элемент которой $b_{ij}$ равен единице тогда и~только тогда, когда среди соседей $a_{ij}$ нет элементов, больших его.
\end{enumerate}

\item
\begin{enumerate}[label=\arabic{enumi}.\arabic*.] % ------- #6 -------
\item
  В~квадратной матрице $A$ порядка~$n$ содержится ровно $n$ единиц, остальные элементы~— нули. Если в~матрице есть хотя бы одна строка или хотя бы один  столбец, состоящий из одних нулей, вывести \textex{0}, иначе вывести \textex{1}.
  \item
  «Тестирование коллектива». Вещественная матрица размера $n × 4$ содержит информацию об студентах некоторой группы из $n$ человек. В~первом столбце содержится масса (в~кг), во втором~— рост (в~см), в~третьем~— средний балл за сессию, в~четвертом~— расстояние места проживания от университета (в~м). Студент называется \emph{среднестатистическим} (\emph{уникальным} по $k$-му параметру) по $k$-му параметру, если его $k$-й парметр минимально (максимально) отличается от среднего арифметического всех чисел из $k$-го столбца. Студент называется \emph{максимально уникальным} (\emph{максимально средним}), если он является уникальным (среднестатистическим) по макисмальному количеству параметров. По данной матрице определить максимально уникальных студентов и~максимально средних. Вывести номера их строк.
\end{enumerate}

\item
\begin{enumerate}[label=\arabic{enumi}.\arabic*.] % ------- #7 -------
  \item
  Дана квадратная матрица порядка~$n$. Замените нулем все ее элементы, расположенные выше главной диагонали. 
  \item
  Из файла \texttt{input.txt} ввести натуральные числа $n$, $m$, $k$, $l$. Гарантируется, что $1\leq k \leq n$, $1\leq l \leq m$. Построить матрицу $A$ размера $n\times m$  следующим образом. Для введенных $k$ и~$l$ элементу $a_{kl}$ присвоить значение~1; элементам, окаймляющим его~— значение~2; элементам следующего окаймления~— значение~3, и~так далее до~заполнения всей матрицы.\\
  Пример ввода:\\\textex{3 4 1 3}\\
  Правильный вывод:\\\textex{
  3 2 1 2\\ 
  3 2 2 2\\ 
  3 3 3 3}
\end{enumerate}

\item
\begin{enumerate}[label=\arabic{enumi}.\arabic*.] % ------- #8 -------
  \item
  Дана квадратная матрица порядка~$n$, заполненная нулями и~единицами. Найдите номер первого по порядку столбца, содержащего наибольшее количество единиц на пересечении с~четными строками (считая строки с~1).
  \item
  Матрица $A$ размера $n \times m$ состоит из нулей и~единиц. Какова в~ней длина самой длинной цепочки подряд стоящих нулевых элементов по горизонтали, вертикали и~диагоналям (главной и~побочной)?
\end{enumerate}

\item
\begin{enumerate}[label=\arabic{enumi}.\arabic*.] % ------- #9 -------
  \item
  Дана действительная матрица размера $n\times m$. Вывести числа $b_1, b_2,  \ldots,  b_n$, равные соответственно суммам элементов строк. 
  \item
  Даны две квадратных матрицы $A$ и~$B$ порядка~$n$. Если $B$ можно получить из~$A$ применением операции транспонирования относительно главной или побочной диагоналей, вывести~\textex{1}, иначе~\textex{0}.
\end{enumerate}

\item
\begin{enumerate}[label=\arabic{enumi}.\arabic*.] % ------- #10 -------
  \item
  Дана вещественная матрица размера $n × m$. Получите числа $b_1, b_2, \ldots b_n$, где $b_i$~— это значение первого по порядку отрицательного элемента $i$-й строки (если таких элементов нет, то~$b_i = 0$).
  \item
  Дана квадратная вещественная матрица порядка~$n$, все элементы которой различны. Найти скалярное произведение строки, в~которой находится наибольший элемент матрицы, на столбец, содержащий наименьший элемент.
\end{enumerate}

\item
\begin{enumerate}[label=\arabic{enumi}.\arabic*.] % ------- #11 -------
  \item
  Дана действительная матрица размера $n\times m$. Определить числа $b_1, b_2, \ldots b_n$, равные соответственно произведениям всех элементов каждой из строк матрицы.
  \item
  Дана квадратная целочисленная матрица порядка~$n$ и~натуральное число $k$. Сформировать одномерный массив, элементами которого являются построчные суммы, но только для тех строк, которые начинаются с~$k$ идущих подряд положительных чисел.
\end{enumerate}

\item
\begin{enumerate}[label=\arabic{enumi}.\arabic*.] % ------- #12 -------
  \item
  Дана действительная матрица размера $n × m$. Определить числа $b_1, b_2, \ldots b_m$, равные соответственно минимальным значениям в~каждом столбце.
  \item
  Матрица $A$ размера $n × m$ ($m$ четное) разделена пополам на левую и~правую половину. Вывести суммы элементов каждого столбца левой половины и~суммы элементов каждой строки (длины $\frac{m}{2}$) правой половины.
\end{enumerate}

\item
\begin{enumerate}[label=\arabic{enumi}.\arabic*.] % ------- #13 -------
  \item
  Дана действительная матрица размера $n\times m$. Определить числа $b_1, b_2, \ldots b_n$, равные соответственно значениям средних арифметических элементов каждой строки матрицы.
  \item
  Дана матрица $A$ размера $n\times m$ и~целые числа $b$, $c$. Определить номера строк матрицы, хотя бы один элемент которых равен $b$, и~увеличить элементы этих строк в~$c$~раз.
\end{enumerate}

\item
\begin{enumerate}[label=\arabic{enumi}.\arabic*.] % ------- #14 -------
  \item
  Дана действительная матрица размера $n\times m$. Определить числа $b_1, b_2, \ldots b_n$, равные соответственно суммам максимальных и~минимальных значений элементов каждой строки матрицы.
  \item
  Дана вещественная матрица $A$ размера $n\times m$ и~натуральное число $k$. Расположить столбцы матрицы в~порядке возрастания элементов в~$k$-й строке. 
\end{enumerate}

\item
\begin{enumerate}[label=\arabic{enumi}.\arabic*.] % ------- #15 -------
  \item
  Дана действительная квадратная матрица порядка~$n$. Заменить нулями все ее элементы, расположенные на побочной диагонали и~ниже нее.
  \item
  Определить минимальный элемент каждой четной строки матрицы размера $n\times m$.
\end{enumerate}

\item
\begin{enumerate}[label=\arabic{enumi}.\arabic*.] % ------- #16 -------
  \item
  Дана действительная квадратная матрица порядка~$n$. Заменить нулями все ее элементы, расположенные ниже главной диагонали.
  \item
  Дана целочисленная матрица и~целые числа $A$, $B$ ($A<B$). Вывести номера тех строк, в~которых все элементы лежат в~отрезке $[A, B]$, если таковых нет, вывести одно число \textex{-1}.
\end{enumerate}

\item
\begin{enumerate}[label=\arabic{enumi}.\arabic*.] % ------- #17 -------
  \item
  Найти количество и~сумму положительных элементов квадратной матрицы порядка~$n$, находящихся выше главной диагонали.
  \item
  Пусть дана действительная матрица размера $n\times m$. Преобразовать матрицу: поэлементно вычесть последнюю строку из всех строк, кроме последней.
\end{enumerate}

\item
\begin{enumerate}[label=\arabic{enumi}.\arabic*.] % ------- #18 -------
  \item
  Дана квадратная матрица порядка~$n$. Преобразовать матрицу по правилу: строку с~номером $n$ сделать столбцом с~номером $n$, а столбец с~номером $n$~— строкой с~номером $n$.
  \item
  Переставить в~матрице размера $n\times m$ строки следующим образом: первую с~последней, вторую~— с~предпоследней и~так далее.
\end{enumerate}

\item
\begin{enumerate}[label=\arabic{enumi}.\arabic*.] % ------- #19 -------
  \item
  Задана квадратная матрица. Переставить строку с~максимальным элементом на главной диагонали со строкой с~заданным номером m.
  \item
  В~данной действительной квадратной матрице порядка $n$ найти первый максимальный по модулю элемент (в порядке «слева направо и сверху вниз». Построить новую квадратную матрицу порядка $n-1$ путем отбрасывания из исходной матрицы строки и~столбца, на пересечении которых расположен найденный элементов.
\end{enumerate}

\item
\begin{enumerate}[label=\arabic{enumi}.\arabic*.] % ------- #20 -------
  \item
  Дана квадратная матрица $n$-го порядка. Определить, является ли она магическим квадратом, т.\,е. такой, в~которой суммы элементов во всех строках и~столбцах одинаковы. Вывести \textex{1}, если да, или \texttt{0}, если нет.
  \item
  В~данной квадратной матрице порядка~$n$ найти сумму элементов первой из строк, в~которой расположен элемент с~минимальным значением.
\end{enumerate}

\item
\begin{enumerate}[label=\arabic{enumi}.\arabic*.] % ------- #21 -------
  \item
  Элемент матрицы назовем \emph{седловой точкой}, если он является минимальным в~своей строке и~одновременно максимальным в~своем столбце или, наоборот,  максимальным в~своей строке и~минимальным в~своем столбце. Для заданной  матрицы размера $n\times m$ напечатать индексы всех ее седловых точек, каждую пару индексов через пробел в~отдельной строке. Строки и~столбцы нумеруются с~единицы.
  \item
  Дана прямоугольная матрица размера $n × m$, все элементы которой попарно различны. Используя минимальное число перестановок ее строк или столбцов, добейтесь того, чтобы максимальный элемент матрицы оказался в~верхнем левом углу.
\end{enumerate}

\item
\begin{enumerate}[label=\arabic{enumi}.\arabic*.] % ------- #22 -------
\item
  Дана целочисленная квадратная матрица порядка~$n$. Выведите номера строк, все элементы которых четны, если таковых нет, вывести одно число \textex{-1}.
  \item
  Дана прямоугольная матрица размера $n × m$. Найти первую строку с~максимальной и~первую строку с~минимальной суммой элементов. Вывести на печать найденные строки и~суммы их элементов.
\end{enumerate}

\item
\begin{enumerate}[label=\arabic{enumi}.\arabic*.] % ------- #23 -------
  \item
  Дана матрица размера $n × m$. Вывести на печать строку с~максималной отрицательной суммой. Если таковых нет, вывести одно число~\textex{-1} 
  \item
  Дана прямоугольная матрица, все элементы которой попарно различны. Найти наибольший и~наименьший элементы и~поменять их местами.
\end{enumerate}

\item
\begin{enumerate}[label=\arabic{enumi}.\arabic*.] % ------- #24 -------
\item
  Определить, является ли заданная квадратная матрица $n$-го порядка симметричной (относительно главной диагонали).
  \item
  Дана действительная квадратная матрица порядка~$n$, где $n=2k$ четно. Получить новую матрицу, переставляя ее блоки размера $k\times k$ крест-накрест.
\end{enumerate}

\item
\begin{enumerate}[label=\arabic{enumi}.\arabic*.] % ------- #25 -------
\item
  Дана квадратная матрица. Найти в~каждой строке первый максимальный элемент и~поменять его местами с~элементом на главной диагонали в~этой строке.
  \item
  Дана квадратная матрица и~натуральное число $k$. Вывести через пробел количество элементов, кратных $k$, и~наибольший из таких элементов. Если ни один элемент матрицы не кратен $k$, вывести \texttt{0 0}.
\end{enumerate}

\item
\begin{enumerate}[label=\arabic{enumi}.\arabic*.] % ------- #26 -------
  \item
  Дана матрица размера $n × m$. Найти максимальный по модулю элемент матрицы, первый в~порядке «слева направо, сверху вниз». Используя минимальное число перестановок строк или столбцов матрицы, добейтесь того, чтобы этот элемент переместился в~правый верхний угол матрицы.
  \item
  Дана вещественная матрица $A$ порядка~$n$. Будем называть \emph{соседями} элемента матрицы $a_{ij}$ элементы матрицы $A$, у которых номер строки и~номер столбца отличаются, соответственно, от $i$ и~$j$ не более чем на единицу. Постройте матрицу $B$, состоящую из нулей и~единиц, элемент которой $b_{ij}$ равен единице тогда и~только тогда, когда все соседи $a_{ij}$ строго меньше него.
\end{enumerate}

\item
\begin{enumerate}[label=\arabic{enumi}.\arabic*.] % ------- #27 -------
  \item
  Дана квадратная матрица порядка~$n$. Записать на место ее отрицательных элементов матрицы нули, а на место неотрицательных~— единицы. 
  \item
  Задана вещественная квадратная матрица порядка~$n$ и~число $k$. Разделить элементы $k$-й строки на диагональный элемент в~этой строке (гарантируется, что он ненулевой).
\end{enumerate}

\item
\begin{enumerate}[label=\arabic{enumi}.\arabic*.] % ------- #28 -------
\item
  Из файла \texttt{input.txt} ввести натуральное число~$n$. Построить квадратную матрицу порядка~$n$, заполненную следующим образом (пример для $n=5$:
  \[
  \left(
  \begin{array}{ccccc}
  1 & 2 & 3 &  4 &  5\\
  10 & 9 & 8 &  7 &  6\\
  11 & 12 & 13 &  14 &  15\\
  20 & 19 & 18 &  17 &  16\\
  21 & 22 & 23 &  24 &  25
  \end{array}
  \right).
  \]
  \item
  Дана действительная матрица размера $n × m$, все элементы которой различны. В~каждой строке выбирается элемент с~наименьшим значением, затем среди этих чисел выбирается наибольшее. Указать индексы элемента с~найденным значением.
\end{enumerate}

\item
\begin{enumerate}[label=\arabic{enumi}.\arabic*.] % ------- #29 -------
  \item
  Задана квадратная матрица. Переставить строку с~максимальным элементом на главной диагонали со~строкой с~заданным номером~$m$.
  \item
  В~данной действительной квадратной матрице порядка~$n$ найти наибольший по модулю элемент. Получить квадратную матрицу порядка $n-1$ путем отбрасывания из исходной матрицы строки и~столбца, на пересечении которых расположен элемент с~найденным значением.
\end{enumerate}

\item
\begin{enumerate}[label=\arabic{enumi}.\arabic*.] % ------- #30 -------
  \item
  Дана целая квадратная матрица $n$-го порядка. Определить, является ли она магическим квадратом, т.\,е. такой, в~которой суммы элементов во всех строках и~столбцах одинаковы.
  \item
  В~данной действительной квадратной матрице порядка~$n$ найти сумму элементов самой верхней строки, в~которой расположен элемент с~минимальным значением.
\end{enumerate}

\end{enumerate}



\end{document}
